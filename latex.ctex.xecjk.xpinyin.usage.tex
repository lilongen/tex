\documentclass{article}
% \usepackage{xeCJK}
\usepackage[UTF8]{ctex}
\usepackage{xpinyin}
\usepackage{xcolor}
% \setCJKmainfont{SimSun}
\title{Title}
\author{李龙恩的实时}
\begin{document}
\maketitle{}
\section{Introduction}
数学、中英文皆可以混排. This is where you will write your content.
\section{Introduction}
This is where you will write your content. \\ \\
$\color{red}{\sqrt[10]{1 + \frac{y}{x^{2} + x + x \times y}}}$ \\
$a^{\phi(n)} \equiv 1 \pmod{n}$
$\sqrt[10]{1 + \frac{y}{x^{2} + x + x \times y}}$ \\
$a^{\phi(n)} \equiv 1 \pmod{n}$

$\sqrt[10]{1 + \frac{y}{x^{2} + x + x \times y}} \textit{要多注意喽}$ \\
$a^{\phi(n)} \equiv 1 \pmod{n}$
$\sqrt[10]{1 + \frac{y}{x^{2} + x + x \times y}}$ \\
$a^{\phi(n)} \equiv 1 \pmod{n}$
$\sqrt[10]{1 + \frac{y}{x^{2} + x + x \times y}}$ \\
$a^{\phi(n)} \equiv 1 \pmod{n}$

$ed \equiv 1 \pmod{\phi(n)}$

\textit{hello world} \\
\textrm{a lots of apples} \\
\textbf{a lots of apples} \\

\begin{pinyinscope}
列位看官:你道此书从何而来?说起根由,虽近荒唐,细按则深有趣味。
待在下将此来历注明,方使阅者\xpinyin{了}{liao3}然不惑。\\
\xpinyin{了}{le1} \\
\color{red}{\pinyin{ni1 men2 hao3 a1}} 
{\color{green}{\xpinyin*{天 门 中 断 楚 江 开}}}

\end{pinyinscope}

\end{document}
