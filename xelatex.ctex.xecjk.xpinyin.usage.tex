\documentclass{article}
%\usepackage{xeCJK}
\usepackage[UTF8]{ctex}

\usepackage{hyperref}
\usepackage{xpinyin}
\usepackage{xcolor}
% \setCJKmainfont{SimSun}
\title{Title}
\author{李龙恩的实时}
\begin{document}
\xpinyinsetup{ratio=0.3}
\xpinyinsetup{format={\color{red}}}

\maketitle{}

\url{https://en.wikibooks.org/wiki/LaTeX/Mathematics#Symbols} \\
\href{https://en.wikibooks.org/wiki/LaTeX/Mathematics#Symbols}{LaTeX/Mathematics}
\section{Introduction}
数学、中英文皆可以混排. This is where you will write your content.
\section{Introduction}
This is where you will write your content. \\ \\
$\color{red}{\sqrt[10]{1 + \frac{y}{x^{2} + x + x \times y}}}$ \\
$a^{\phi(n)} \equiv 1 \pmod{n}$

$\sqrt[10]{1 + \frac{y}{x^{2} + x + x \times y}} \textit{要多注意喽}$ \\
$a^{\phi(n)} \equiv 1 \pmod{n}$


\textit{hello world} \\
\textrm{a lots of apples} \\
\textbf{a lots of apples} \\


\begin{pinyinscope}
静夜思 \\
李白 \\
床前明月光\\
疑是地上霜\\
举头望明月\\
低头思故乡\\
\end{pinyinscope}

\begin{pinyinscope}
列位看官:你道此书从何而来?说起根由,虽近荒唐,细按则深有趣味。
待在下将此来历注明,方使阅者\xpinyin{了}{liao3}然不惑。\\
\xpinyin{了}{le1} \\
\end{pinyinscope}
\color{red}{\pinyin{ni1 men2 hao3 a1}} 
{\color{green}{\xpinyin*{天 门 中 断 楚 江 开}}}

@article{greenwade93,
    author  = "George D. Greenwade",
    title   = "The {C}omprehensive {T}ex {A}rchive {N}etwork ({CTAN})",
    year    = "1993",
    journal = "TUGBoat",
    volume  = "14",
    number  = "3",
    pages   = "342--351"
}

\end{document}


